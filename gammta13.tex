\documentclass{beamer}
\usepackage{beamerthemeshadow}
%\usepackage{natbib}
\usepackage{amssymb,amsmath,amsthm,mathtools,amscd}
%\beamersetuncovermixins{\opaqueness<1>{25}}{\opaqueness<2->{15}}

\useinnertheme{rounded}
\usecolortheme{beetle}
\setbeamercolor{normal text}{bg=white!20} 
\setbeamercolor{title}{bg=blue!10!gray}
\setbeamercolor{block title}{fg=black,bg=blue!10!gray!50!}

\setbeamercolor{section in toc}{fg=black}
%\logo{\includegraphics[height=0.8cm]{TU-LogoSchrift2line.jpg}} 
\setbeamercolor{logo}{bg=white}
\setbeamercolor{itemize item}{fg=gray}

\DeclareMathOperator{\re}{Re}

\newenvironment{recap}{\begin{small}\color{gray}\begin{flushright}}{\end{flushright}\end{small}}
\input{mathComAbb}


\begin{document}
%%% Spaces
\def\Hdf{H_{{df}}}
\def\sdf{_{df}}
\def\Hg{H_{{g}}}
\def\sg{_{g}}

%%% Projectors
\def\PHdf{\mathcal P_{\Hdf}}
\def\PHpdf{\mathcal P_{\Hdf'}}

\title[Decoupling of semi-explicit index-2 DAEs]{
Decoupling and Optimal Control \\ of Semi-explicit Semi-linear PDAEs of Index 2
}


\author{Jan Heiland}
\institute{ GAMM Annual Meeting, Novi Sad}
\date{\today} 

\titlegraphic{\includegraphics[scale=0.009]{TUBerlin_Logo_rot.jpg} }

\frame{\titlepage} 

\frame{
\vfill
\tableofcontents
} 

\section{Optimal Control of Abstract DAEs}
\subsection{Problem Setting}

\frame{
\frametitle{Optimal Control of ADAEs}
We consider the optimal control problem
\begin{equation*}
	\min_{u} \mathcal J(v,u) := \mathcal M(x(T))+\int_0^T  \mathcal K(t,v(t),u(t) \inva t  
\end{equation*}
for function-valued $v$ and $u$ satisfying the abstract DAE
\begin{align}
	\dot v - A(v) - J_1'p &= f \tag{ADAE}\\
	J_2v \phantom{ - J_1'p}&= g\notag
\end{align}
almost everywhere on $(0,T]$.
}


\frame{
\frametitle{Optimality System}
\begin{recap}
	$\mathcal J(v,u) \to \min$, s.t. \\
	$\dot v - A(v) - J_1'p - f = 0 $ and \\
	$J_2v - g = 0$ (ADAE)
\end{recap}
We want to find an optimal solution via the optimality conditions 
\begin{equation*}
	\begin{bmatrix} \text{(ADAE)} \\ \text{ (adADAE) } \\ \text{ Gradient Cond.} \end{bmatrix} \begin{bmatrix} (v,p) \\ (\lambda_v,\lambda_p) \\ u \end{bmatrix} = 0,
\end{equation*}
where 
\begin{itemize}
	\item (adADAE) is the adjoint to (ADAE) wrt. $\mathcal J$ 
	\item and $(\lambda_v,\lambda_p)$ is the adjoint state.
\end{itemize}
}
\subsection{Existence of Solutions}
\frame{
%\frametitle{Basic Assumptions I}
\begin{block}{Assumption 1}
In the neighborhood of an optimal pair $(v^* ,u^* )$, let
\begin{itemize}
	\item[a.)] the involved operators be two times Fr\'echet differentiable,
	\item[b.)] the linearized (ADAE) have a unique solution for any $u$,
	\item[c.)] the (adADAE) have a solution,
	\item[d.)] there be a constant $c$ such that 
$$-\langle h_v,\lambda_v A'_{;vv}(v^*)h_v \rangle + \mathcal J_{;(v,u)^2}(v^*,u^*)[h_v,h_u]^2 \geq c \norm{h_u}^2$$
		for all $(h_v,h_u)$ that solve the linearized (ADAE).
\end{itemize}
\end{block}
\begin{theorem}
	If Assumption 1 holds, then $(v^*,u^*)$ is a locally unique solution to the optimal control problem. This solution can be approached by Newton's method for the nonlinear optimality system. 
\end{theorem}
}

\frame{
\frametitle{Agenda}
\vfill
In this talk, I will address the parts 
\begin{itemize}
	\item[b.)] solvability of the (ADAE) 
	\item[c.)] and the (adADAE)
\end{itemize}
of Assumption 1 by introducing a splitting of ADAEs.
\vfill
This splitting separates the algebraic and differential variables, such that one can read off consistency and smoothness conditions and use standard abstract theory for existence of solutions to evolution equations.
\vfill
}
\section{A Class of Abstract DAEs}
\subsection{Functional Analysis Setting}
\frame{
\frametitle{Functional Analysis Setting}
\begin{itemize}
	\item Banach space $V \subset H$ Hilbert space (dense and continuously embedded)
	\item Riesz isomorphism $j'\colon H \to H'$ identifies $H$ with its dual $H'$
	\item Then, $V\subset H \cong H' \subset V'$
\end{itemize}
\vfill
 We look for $v\in \bigl( (0,T) \to V \bigr)$ with $\dot v(t) \in V'$ and for $p \in \bigl( (0,T) \to Q \bigr)$ that satisfy
	\begin{align*}
		\dot v(t) -A(v(t))-J_1'p(t)&=f(t) \quad\text{ in }V'\text{, a.e. in }(0,T), \\
		-J_2v(t)\phantom{-J_1'p(t)}&=g(t) \quad\text{ in }Q'\text{, a.e. in }(0,T).
	\end{align*}
with $V\subset H \subset V'$ as defined above, and with a Hilbert space $Q$.
}
\frame{
\frametitle{Abstract ADAE}
	\begin{recap}
		$v(t)\in V$, $\dot v(t) \in V'$, $p(t)\in Q$, \\
		$\dot v(t) -A(v(t))-J_1'p(t)=f(t) \quad\text{ in }V'$ \\
		$-J_2v(t)\phantom{-J_1'p(t)}=g(t) \quad\text{ in }Q'$
	\end{recap}
With operators 
\begin{itemize}
	\item $A\colon V\to V'$
	\item $J_1':Q \to V'$
	\item $J_2 :V \to Q'$
\end{itemize}
\vfill
Major obstacle for decoupling: \\
The equations are posed in a Banach space $V' \supsetneq V$
\begin{itemize}
	\item[$\to$] A projection does not necessarily split the space  
	\item[$\to$] Solution space is different from ``equation space''
\end{itemize}
\vfill
}
\subsection{Decoupling of ADAEs}
\frame{
\frametitle{Strategy}
We proceed as follows:
\begin{itemize}
	\item Given a smooth $f$ and smoothing operators,
	\item the equation is posed in $H'$ -- rather than in $V'$ --
	\item which is a Hilbert space.
	\item There we can define projectors and decouple the equations.
	\item $v\in V \subset H$ can be decoupled with the dual projections.
\end{itemize}
}

\frame{
\frametitle{[Well posedness] and [Smoothness] Assumption}
\begin{recap}
	$\dot v(t) -A(v(t))-J_1'p(t)=f(t) \quad\text{ in }V'$ \\
	$-J_2v(t)\phantom{-J_1'p(t)}=g(t) \quad\text{ in }Q'$
\end{recap}
\begin{block}{Assumption [WP]}
	\begin{itemize}
		\item[(a)] $J_1',J_2'\in \mathcal L(Q,H')$ are homeomorphisms onto their range and 
		\item[(b)] $j(\image J_1') \cap \ker J_2=\{0\}$,
	\end{itemize}
	where $j\colon H'\to H$ is the Riesz isomorphism.
\end{block}

\begin{block}{Assumption [S]}
	For more regular data $f(t)\in H'$ (rather than in $V'$) any corresponding solution $(v,p)$ of the abstract DAE invokes $A(v(t))\in H'$.
\end{block}
}


\frame{
\frametitle{Splitting of the ADAE}
\begin{theorem}
	Under Assumptions [WP], we have 
	$$H' = j(\ker J_2) \oplus \image J_1'=: \Hdf' \oplus \Hg',$$
	defining a projector $\PHpdf$.
	If, in addition [S] holds, then, for $f$ and $g$ sufficiently smooth, any $v$ with values in $V$ solving the ADAE
	\begin{align*}
		\dot v(t) -A(v(t))-J_1'p(t)&=f(t) \quad\text{ in }V', \\
		-J_2v(t)\phantom{-J_1'p(t)}&=g(t) \quad\text{ in }Q',
	\end{align*}
	 a.e. in $(0,T)$, can be written as $v=v\sdf + v\sg$, where 
\begin{align*}
	j'v\sg(t) = -J_1'S^{-1}g(t) &\text{ in }H_g'\\
	\dot v\sdf (t)  - \PHpdf A\bigl(v\sdf(t) -jJ_1S^{-1}g(t)\bigr)  = \PHpdf f(t) & \text{ in }\Hdf'.
\end{align*}

\end{theorem}
}


\frame{
\frametitle{What Else?}
Further issues:
\begin{itemize}
	\item Regularity of $\bigl( t \mapsto v(t),p(t) \bigr)$
	\item Initial conditions and consistency
	\item Spatial discretizations
\end{itemize}
}

\frame{
\begin{center}
Thanks to Volker Mehrmann and \\
\vspace{0.1in}
\textbf{thank you for your attention.}\\
\vspace{0.2in}
For suggestions and questions please contact me\\
\vspace{0.1in}
\texttt{heiland@math.tu-berlin.de}
\end{center}
}
\end{document}
