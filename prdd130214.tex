\documentclass{beamer}
\usepackage{beamerthemeshadow}
%\usepackage{natbib}
\usepackage{amssymb,amsmath,amsthm,mathtools,amscd}
%\beamersetuncovermixins{\opaqueness<1>{25}}{\opaqueness<2->{15}}

\useinnertheme{rounded}
\usecolortheme{beetle}
\setbeamercolor{normal text}{bg=white!20} 
\setbeamercolor{title}{bg=blue!10!gray}
\setbeamercolor{block title}{fg=black,bg=blue!10!gray!50!}

\setbeamercolor{section in toc}{fg=black}
%\logo{\includegraphics[height=0.8cm]{TU-LogoSchrift2line.jpg}} 
\setbeamercolor{logo}{bg=white}
\setbeamercolor{itemize item}{fg=gray}

\DeclareMathOperator{\re}{Re}

\newenvironment{recap}{\begin{small}\color{gray}\begin{flushright}}{\end{flushright}\end{small}}
\input{mathComAbb}


\begin{document}
%%% Spaces
\def\Hdf{H_{{df}}}
\def\sdf{_{df}}
\def\Hg{H_{{g}}}
\def\sg{_{g}}

%%% Projectors
\def\PHdf{\mathcal P_{\Hdf}}
\def\PHpdf{\mathcal P_{\Hdf'}}

\title[Decoupling of semi-explicit index-2 DAEs]{ Decoupling of Semi-Explicit Index-2 (A)DAEs  \\ via Projector Chains}


\author{Jan Heiland}
\institute{ Absolventen-Seminar Numerische Mathematik }
\date{\today} 

\titlegraphic{\includegraphics[scale=0.009]{TUBerlin_Logo_rot.jpg} }

\frame{\titlepage} 

\frame{
\vfill
\tableofcontents
} 

\section{Finite Dim DAEs}
\subsection{No Introduction}

\frame{
\frametitle{Finite-dimensional NSEs}
Differential-algebraic equation (DAE) for $v\in \mathcal C^1(0,T;\mathbb R^{n_v})$ and $p\in \mathcal C(0,T;\mathbb R^{n_v})$ 
\begin{equation*}
	\bbmat M & 0 \\ 0 & 0 \ebmat\bbmat \dot v \\ \dot p \ebmat-\bbmat A(v) & J_1^T \\ J_2 & 0 \ebmat\bbmat v \\ p \ebmat=\bbmat f \\ g \ebmat
\end{equation*}
 in  $\mathcal C^1(0,T;\mathbb R^{n_v}) \times \mathcal C(0,T;\mathbb R^{n_v}).$
\vfill
Crucial:
\begin{itemize}
	\item $M$, $J_2M^{-1}J_1^T$ invertible ($\Rightarrow n_p \leq n_v$)
\end{itemize}
For simplification:
\begin{itemize}
	\item $M$, $J_1$, $J_2$ are time-invariant
\end{itemize}
}

\frame{
\frametitle{Projector Chain}
\begin{itemize}
	\item Start from the DAE $(E_0,A_0)$, ie. $E_0\dot x - A_0x = f$
\end{itemize}
Given $(E_k,A_k)$\dots
\begin{enumerate}
	\item \texttt{If} $E_k$ is invertible, \texttt{Return}
	\item \texttt{Elseif}:
		\begin{itemize}
			\item Compute projector $Q _k$, with $\image Q_k = \ker E_k$
			\item Define $P_k:= I - Q_k$
			\item Set $(E_{k+1}, A_{k+1}):= (E_k+A_kQ_k,A_kP_k)$
			\item Dump $k$ and start over
		\end{itemize}
\end{enumerate}
The exiting $k$ defines the tractability index of the DAE
}
\subsection{Semidiscrete Navier-Stokes}
\frame{
\frametitle{A Concrete Example}
\begin{recap}
	$\bbmat M & 0 \\ 0 & 0 \ebmat\bbmat \dot v \\ \dot p \ebmat-\bbmat A(v) & J_1^T \\ J_2 & 0 \ebmat\bbmat v \\ p \ebmat=\bbmat f \\ g \ebmat$ \\
	as $\mathcal E_0\dot x - \mathcal A_0x = [fg] $
\end{recap}
For the NSE system we find that $\mathcal E_2$ is invertible with
\begin{equation*}
	\mathcal E_2^{-1} = \begin{bmatrix} \mathcal P & [I-\mathcal PM^{-1}A]M^{-1}J_1^TS^{-1} \\ {\mathcal Q}^- & -[I+{\mathcal Q}^-M^{-1}AM^{-1}J_1^T]S^{-1} \end{bmatrix}
\end{equation*}
with $S=J_2M^{-1}J_1^T$, the projectors 
\begin{itemize}
	\item[] ${\mathcal Q}=M^{-1}J_1^TS^{-1}J_2$ and 
	\item[] $\mathcal P = I - \mathcal Q$,
\end{itemize} 
and for any $A=A(v)$.
}

\frame{
\frametitle{and Application}
We scale the NSE system by $\mathcal E_2^{-1}$, write $v=\mathcal Pv + \mathcal Qv$, and end up with the decoupling
\begin{align*}
	\mathcal Q v &= - M^{-1}J_1^TS^{-1}g \\
	\frac{d}{dt}(\mathcal P v) - \mathcal PM^{-1}A(\mathcal Pv+\mathcal Qv)[\mathcal Pv+\mathcal Qv] &= \mathcal PM^{-1}f, \\
	p &= \mathcal F(\mathcal Pv,\mathcal Q_v,\dot g).
\end{align*}
Note that $\mathcal  P = I - M^{-1}J_1^TS^{-1}J_2$ projects onto the kernel of $J_2$.
\vfill

This decoupling works, since $\mathcal P$ and $\mathcal Q$ split the solution space as well as the ``equation space''.
}

\section{Abstract DAEs}
\subsection{General Considerations}
\frame{
\frametitle{The Main Result}
A similar result for abstract DAEs, where $v$ and $p$ take on values in Banach-spaces, that will
\begin{itemize}
	\item Decouple algebraic an differential equations,
	\item Characterise the solution sets, 
	\item Reveal consistency and smoothness conditions (not today),
	\item and that is consistent with finite-dimensional solution approaches.
\end{itemize}
}

\frame{
\frametitle{Functional Analysis Setting}
\begin{itemize}
	\item Banach space $V \subset H$ Hilbert space (dense and continuously embedded)
	\item Riesz isomorphism $j'\colon H \to H'$ identifies $H$ with its dual $H'$
	\item Then, $V\subset H \cong H' \subset V'$
\end{itemize}
\vfill
 We look for $v\in \bigl( (0,T) \to V \bigr)$ with $\dot v(t) \in V'$ and for $p \in \bigl( (0,T) \to Q \bigr)$ that satisfy
	\begin{align*}
		\dot v(t) -A(v(t))-J_1'p(t)&=f(t) \quad\text{ in }V'\text{, a.e. in }(0,T), \\
		-J_2v(t)\phantom{-J_1'p(t)}&=g(t) \quad\text{ in }Q'\text{, a.e. in }(0,T).
	\end{align*}
with $V\subset H \subset V'$ as defined above, and with a Hilbert space $Q$.
}
\frame{
\frametitle{Abstract ADAE}
	\begin{recap}
		$v(t)\in V$, $\dot v(t) \in V'$, $p(t)\in Q$, \\
		$\dot v(t) -A(v(t))-J_1'p(t)=f(t) \quad\text{ in }V'$ \\
		$-J_2v(t)\phantom{-J_1'p(t)}=g(t) \quad\text{ in }Q'$
	\end{recap}
With operators 
\begin{itemize}
	\item $A\colon V\to V'$
	\item $J_1':Q \to V'$
	\item $J_2 :V \to Q'$
\end{itemize}
\vfill
Major obstacle for decoupling: \\
The equations are posed in a Banach space $V' \supsetneq V$
\begin{itemize}
	\item[$\to$] A projection does not necessarily split the space  
	\item[$\to$] Solution space is different from ``equation space''
\end{itemize}
\vfill
}

\frame{
\frametitle{Strategy}
We proceed as follows:
\begin{itemize}
	\item Given a smooth $f$ and smoothing operators,
	\item the equation is posed in $H'$ -- rather than in $V'$ --
	\item which is a Hilbert space.
	\item There we can define projectors and decouple the equations.
	\item $v\in V \subset H$ can be decoupled with the dual projections.
\end{itemize}
}

\frame{
\frametitle{(Well posedness) Assumption}
\begin{recap}
	$\dot v(t) -A(v(t))-J_1'p(t)=f(t) \quad\text{ in }V'$ \\
	$-J_2v(t)\phantom{-J_1'p(t)}=g(t) \quad\text{ in }Q'$
\end{recap}
\begin{block}{Assumption [WP]}
	\begin{itemize}
		\item[(a)] $J_1',J_2'\in \mathcal L(Q,H')$ are homeomorphisms onto their range and 
		\item[(b)] $j(\image J_1') \cap \ker J_2=\{0\}$,
	\end{itemize}
	where $j\colon H'\to H$ is the Riesz isomorphism.
\end{block}

\begin{lemma}
	If $J_1=J_2=:J$, then Assumption [WP] is equivalent to the commonly used condition
		\begin{equation*}
			\exists \gamma >0: \inf_{0\neq v\in H} \sup_{0\neq q \in Q} \frac{\langle J'q,v \rangle _{H',H}}{\norm{q}_Q\norm{v}_H}\geq \gamma > 0,
		\end{equation*}
\end{lemma}
}

\frame{
\frametitle{(Smoothness) Assumption}
\begin{recap}
	$\dot v(t) -A(v(t))-J_1'p(t)=f(t) \quad\text{ in }V'$ \\
	$-J_2v(t)\phantom{-J_1'p(t)}=g(t) \quad\text{ in }Q'$
\end{recap}

\begin{block}{Assumption [S]}
	For more regular data $f(t)\in H'$ (rather than in $V'$) any corresponding solution $(v,p)$ of the abstract DAE invokes $A(v(t))\in H'$.
\end{block}

\vfill

\begin{corollary}
With $J_1'\colon Q \to H'$, Assumption [S] means that for smoother $f$ the $ADAE$ is posed in the Hilbert space $H'\times Q'$
\end{corollary}
}


\frame{
\frametitle{Split the Spaces}
	By Assumption we have $S:=J_2jJ_1' \colon Q \to Q'$ is invertible, and we can define $L:=J_1'S^{-1}J_2\colon H \to H'$.
\begin{lemma}
\begin{itemize}
		\item $\Hdf:= \ker J_2=\image [I_{H}-jL]$,
		\item $\Hg:= \image jL = \image jJ_1'$,
		\item[and] thus $H=\Hdf\oplus \Hg$,
		\item $\Hg':= \image Lj = j'(\Hg)$,
		\item $\Hdf':= \image [I_{H'}-Lj]=j'(\Hdf)$,
		\item[and] thus $H'=\Hdf'\oplus \Hg'$.
\end{itemize}
\end{lemma}
Define $\PHpdf$, $\PHdf$ -- corresponding projectors onto $\Hdf'$ and $\Hdf$.
}

\frame{
\frametitle{Decouple the Equations}
\begin{recap}
	For smooth $f$ we have \\
	$\dot v(t) -A(v(t))-J_1'p(t)=f(t) \quad\text{ in }H'$ \\
	$-J_2v(t)\phantom{-J_1'p(t)}=g(t) \quad\text{ in }Q'$
\end{recap}
And, having scaled the ADAE by the injective 
 \begin{equation*}
	 \mathcal E_2^{-1}:=\bbmat \PHpdf  & J_1'S^{-1} \\ j_Q'S^{-1}J_2j & -j_Q'S^{-1} \ebmat\colon H'\times Q' \to H'\times Q' . 
 \end{equation*}
(from the projector chain), we get the equivalent system of equation
$$ \PHpdf \dot v  - \PHpdf A(v)-Lv  = \PHpdf f + J_1S^{-1}g \text{ in }H'.$$
plus an equation defining the algebraic variable $p$.
}

\frame{
\frametitle{Split the Equations}
\begin{recap}
$ \PHpdf \dot v  - \PHpdf A(v)-Lv  = \PHpdf f + J_1'S^{-1}g \text{ in }H'$
\end{recap}
Using $H'=\Hdf'\oplus \Hg'$ and $v= \PHdf v+jLv $ we find
that
\begin{equation*}
	Lv = [I-\PHdf]v = -J_1S^{-1}g \text{ in }H_g'
\end{equation*}
while $\PHdf v$ must solve 
\begin{equation*}
	\dot w  - \PHpdf A(w -jJ_1S^{-1}g)  = \PHpdf f  \text{ in }\Hdf'.
\end{equation*}
}

\frame{
\frametitle{Summing Up}
\begin{theorem}
	Under Assumptions [WP] and [S], for $f$ and $g$ sufficiently smooth,
	any $v$ with values in $V$ solving the ADAE
	\begin{align*}
		\dot v(t) -A(v(t))-J_1'p(t)&=f(t) \quad\text{ in }V', \\
		-J_2v(t)\phantom{-J_1'p(t)}&=g(t) \quad\text{ in }Q',
	\end{align*}
	 a.e. in $(0,T)$, can be written as $v=\PHdf v + jLv$, where the parts solve 
\begin{align*}
	Lv(t) = -J_1S^{-1}g(t) &\text{ in }H_g'\\
	\dot \PHdf v(t)  - \PHpdf A\bigl(\PHdf v(t) -jJ_1S^{-1}g(t)\bigr)  = \PHpdf f(t) & \text{ in }\Hdf'.
\end{align*}

\end{theorem}
}

\subsection{Navier-Stokes Case}
\frame{
\frametitle{NSE vs. the Assumptions }
\begin{recap}
	$\dot v(t) -A(v(t))-J_1'p(t)=f(t) \quad\text{ in }V'$ \\
	$-J_2v(t)\phantom{-J_1'p(t)}=g(t) \quad\text{ in }Q'$
\end{recap}
\begin{block}{Assumption [WP]}
	\begin{itemize}
		\item[(a)] $J_1',J_2'\in \mathcal L(Q,H')$ are homeomorphisms onto their range and 
		\item[(b)] $j(\image J_1') \cap \ker J_2=\{0\}$,
	\end{itemize}
	where $j\colon H'\to H$ is the Riesz isomorphism.
\end{block}

\begin{block}{Assumption [S]}
	For more regular data $f(t)\in H'$ (rather than in $V'$) any corresponding solution $(v,p)$ of the abstract DAE invokes $A(v(t))\in H'$.
\end{block}

}



\frame{
\frametitle{What Else?}
Further issues:
\begin{itemize}
	\item Regularity of $\bigl( t \mapsto v(t),p(t) \bigr)$
	\item Initial conditions and consistency
	\item Spatial discretizations
	\item Optimal control problems
\end{itemize}
}

\frame{
\begin{center}
Thanks to Volker Mehrmann and \\
\vspace{0.1in}
\textbf{thank you for your attention.}\\
\vspace{0.2in}
For suggestions and questions please contact me\\
\vspace{0.1in}
\texttt{heiland@math.tu-berlin.de}
\end{center}
}
\end{document}
